\begin{thebibliography}{99}
\bibitem{DSTU_2394} \textit{ДСТУ 2394-94} Інформація та документація. Комплектування фонду,бібліографічний опис, аналіз документів. Терміни та визначення. — Чинний від
01.01.1995. — Київ: Держстандарт України, 1994. — 88 с.
\bibitem{DSTU_5034:2008} \textit{ДСТУ 5034:2008 } Інформація і документація. Науково-інформаційна діяльність.
Терміни та визначення понять. — Київ: Держспоживстандарт України, 2009. — 38 с.
\bibitem{DSTU_3008-95} \textit{ДСТУ 3008-95} Документація.
Звіти у сфері науки і техніки.Структура і правила оформлення.— Київ: Держспоживстандарт України,1995. - 39 c.
\bibitem{Frike_DE} \textit{Фрике К.} Вводный курс цифровой электроники — Москва:
Техносфера,2003. - 432 с. ISBN 5-94836-015-6
\bibitem{Tanenbaum_CA} \textit{Таненбаум Э.} Архитектура компьютера. 5-е изд. (+CD). — СПб.: Питер, 2007. — 844 с: ил. ISBN 5-469-01274-3 
\bibitem{metodichka}\textit{Торошанко Я.} Методичні рекомендації до курсового проекту. -
Київ : ДУТ, 2018. - 4 с.
\bibitem{Vernadskii}\textit{Національна бібліотека України імені В. І. Вернадського} [Електронний
ресурс]: [Веб-сайт]. – Електронні дані. – Київ : НБУВ, 2013-2017. – Режим доступу:
www.nbuv.gov.ua дата звернення 30.03.2017) – Назва з екрана.
\end{thebibliography}
\begin{center}
\textsc{МІНІСТЕРСТВО ОСВІТИ І НАУКИ УКРАЇНИ \\[2mm]
ДЕРЖАВНИЙ УНІВЕРСИТЕТ ТЕЛЕКОМУНІКАЦІЙ \\[5mm]
НАВЧАЛЬНО-НАУКОВИЙ ІНСТИТУТ ТЕЛЕКОМУНІКАЦІЙ ТА ІНФОРМАТИЗАЦІЇ\\[2mm]
Кафедра комп'ютерної інженерії}

\vfill

\textbf{ЗАВДАННЯ \\[3mm]
на курсовий проект студенту\\[6mm]
Максімов Євгеній Сергійович\\[6mm]
}
\end{center}
\begin{enumerate}
	\item Тема проекту:
	\begin{enumerate}[label={1.\arabic*}]
		\item Мінімізація та побудова логічної функції.
		\item Побудова багатовходового дешифратора.
	\end{enumerate}
	\item Вхідні дані проекту:
	\begin{enumerate}[label={2.\arabic*}]
		\item Таблиця істинності 2-вох логічних функцій згідно варіанту.
		\item Таблиця обмежень побудови неповного дешифратора відповідно до варіанту для двох адресних просторів А1 і А2.
	\end{enumerate}
	\item Зміст розрахунково-пояснювальної записки (перелік питань, які потрібно
розробити):
	\begin{enumerate}[label={3.\arabic*}]
		\item  Перша логічна функція.
		\begin{enumerate}[label={3.1.\arabic*}]
			\item Мінімізація та побудова схеми функції.
			\item Переведення в базис І-НЕ(NAND) та побудова схеми функції.
			\item Переведення в базис АБО-НЕ(NOR) та побудова схеми функції.
		\end{enumerate}
		\item  Друга логічна функція.
		\begin{enumerate}[label={3.2.\arabic*}]
			\item Мінімізація та побудова схеми функцій.
			\item Переведення в базис І-НЕ(NAND) та побудова схеми функції.
			\item Переведення в базис АБО-НЕ(NOR) та побудова схеми функції.
		\end{enumerate}
		\item Побудова багатовходового дешифратора.
		\begin{enumerate}[label={3.3.\arabic*}]
			\item Побудувати схему неповного дешифратора відповідно до варіанту для двох адресних просторів А1 і А2.
			\item Привести  таблицю, в якій для кожної мікросхеми останнього ступеню вказати її адресний простір.
			\item Оцінити апаратні витрати на побудову дешифратора.
		\end{enumerate}
	\end{enumerate}
\end{enumerate}
\newpage
\begin{center}
\textbf{КАЛЕНДАРНИЙ ПЛАН\\[2mm]}
\begin{tabu} { | X[-2.5,с] | X[3,c] | X[0,c] | X[0,c] | }
 \hline
 № \newline з/п & Назва етапів виконання курсового проекту & Строк виконання етапів роботи & Примітка\\
 \hline
 1 & Підбір науково-технічної літератури &  &  \\
\hline
 2 & Проведення необхідних обчислень &  &  \\
\hline
 3 & Розробка  креслень&  &  \\
\hline
 4 & Розробка розрахунково-пояснювальної записки та реферату&  &  \\
\hline
\end{tabu}
\end{center}
\vspace{6mm}\\
\begin{flushright}
Студент
\begin{minipage}[t][2\mytextsize][t]{2in} % размер minipage равен удвоенному размеру основного шрифта
\underline{\hspace{2in}}\\ % линия подчёркивания на два дюйма
  \centering
  \small(підпис)
  \vspace{\mytextsize} % отступ minipage для выравнивания линии подчёркивания с базовой линией остального текста
\end{minipage} 
\begin{minipage}[t][2\mytextsize][t]{2in} % размер minipage равен удвоенному размеру основного шрифта
\underline{\hspace{2in}}\\ % линия подчёркивания на два дюйма
  \centering
  \small(прізвище та ініціали)
  \vspace{\mytextsize} % отступ minipage для выравнивания линии подчёркивания с базовой линией остального текста
\end{minipage}
\vspace{10mm}\\
Керівник роботи
\begin{minipage}[t][2\mytextsize][t]{2in} % размер minipage равен удвоенному размеру основного шрифта
\underline{\hspace{2in}}\\ % линия подчёркивания на два дюйма
  \centering
  \small(підпис)
  \vspace{\mytextsize} % отступ minipage для выравнивания линии подчёркивания с базовой линией остального текста
\end{minipage} 
\begin{minipage}[t][2\mytextsize][t]{2in} % размер minipage равен удвоенному размеру основного шрифта
\underline{\hspace{2in}}\\ % линия подчёркивания на два дюйма
  \centering
  \small(прізвище та ініціали)
  \vspace{\mytextsize} % отступ minipage для выравнивания линии подчёркивания с базовой линией остального текста
\end{minipage}
\end{flushright}
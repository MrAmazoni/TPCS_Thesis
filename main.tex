\documentclass[14pt,a4paper,report]{ncc}
\usepackage[a4paper, mag=1000, left=2.5cm, right=1cm, top=2cm, bottom=2cm, headsep=0.7cm, footskip=1cm]{geometry}
\usepackage[utf8]{inputenc}
\usepackage[english,russian]{babel}
\usepackage{indentfirst}
\usepackage[dvipsnames]{xcolor}
\usepackage[colorlinks]{hyperref}
\usepackage{listings} 
\usepackage{caption}
\usepackage{enumitem}
\usepackage{tabu}
\usepackage{cmap}
%page numbering
\usepackage{scrpage2}
\ifoot[]{}
\cfoot[]{}
\ofoot[\pagemark]{\pagemark}
\pagestyle{scrplain}
%centered contensts
%=======================================================================
\usepackage[tocflat]{tocstyle}
\usetocstyle{standard}

% Redefinition of ToC command to get centered heading
\makeatletter
\renewcommand\tableofcontents{%
  \null\hfill\textbf{\Large\contentsname}\hfill\null\par
  \@mkboth{\MakeUppercase\contentsname}{\MakeUppercase\contentsname}%
  \@starttoc{toc}%
}
%========================================================================
\usepackage{graphicx}
\graphicspath{ {./} }
\DeclareCaptionFont{white}{\color{white}} 
\DeclareCaptionFormat{listing}{\colorbox{gray}{\parbox{\textwidth}{#1#2#3}}}
\captionsetup[lstlisting]{format=listing,labelfont=white,textfont=white}
\lstset{% Собственно настройки вида листинга
inputencoding=utf8, extendedchars=\true, keepspaces = true, % поддержка кириллицы и пробелов в комментариях
language=C++,            % выбор языка для подсветки (здесь это Pascal)
basicstyle=\small\sffamily, % размер и начертание шрифта для подсветки кода
numbers=left,               % где поставить нумерацию строк (слева\справа)
numberstyle=\tiny,          % размер шрифта для номеров строк
stepnumber=1,               % размер шага между двумя номерами строк
numbersep=5pt,              % как далеко отстоят номера строк от подсвечиваемого кода
backgroundcolor=\color{white}, % цвет фона подсветки - используем \usepackage{color}
showspaces=false,           % показывать или нет пробелы специальными отступами
showstringspaces=false,     % показывать или нет пробелы в строках
showtabs=false,             % показывать или нет табуляцию в строках
frame=single,               % рисовать рамку вокруг кода
tabsize=2,                  % размер табуляции по умолчанию равен 2 пробелам
captionpos=t,               % позиция заголовка вверху [t] или внизу [b] 
breaklines=true,            % автоматически переносить строки (да\нет)
breakatwhitespace=false,    % переносить строки только если есть пробел
escapeinside={\%*}{*)}      % если нужно добавить комментарии в коде
}

\begin{document}
% Переоформление некоторых стандартных названий
\def\contentsname{ЗМІСТ}
\renewcommand{\chaptername}{Розділ}
\renewcommand{\bibname}{Список джерел}
% Оформление титульного листа
\begin{titlepage}
\begin{center}
\textsc{МІНІСТЕРСТВО ОСВІТИ І НАУКИ УКРАЇНИ \\[2mm]
ДЕРЖАВНИЙ УНІВЕРСИТЕТ ТЕЛЕКОМУНІКАЦІЙ \\[5mm]
НАВЧАЛЬНО-НАУКОВИЙ ІНСТИТУТ ТЕЛЕКОМУНІКАЦІЙ ТА ІНФОРМАТИЗАЦІЇ\\[2mm]
Кафедра комп'ютерної інженерії}

\vfill

\textbf{РОЗРАХУНКОВО-ПОЯСНЮВАЛЬНА ЗАПИСКА \\[3mm]
до курсового проекту з дисципліни «Технології проектування ком'ютерних систем»\\[6mm]
Варіант 13
\\[20mm]
}
\end{center}

\hfill
\begin{minipage}{.5\textwidth}
Виконав студент:\\[2mm] 
Максімов Євгеній Сергійович\\
група: КІД-31\\[5mm]

Керівник роботи:\\[2mm] 
кандидат технічних наук, доцент\\
Торошанко Ярослав Іванович 
\end{minipage}%
\vfill
\begin{center}
 Київ, \theyear\ г.
\end{center}
\end{titlepage}

% Содержание
\newpage
\begin{center}
\textsc{МІНІСТЕРСТВО ОСВІТИ І НАУКИ УКРАЇНИ \\[2mm]
ДЕРЖАВНИЙ УНІВЕРСИТЕТ ТЕЛЕКОМУНІКАЦІЙ \\[5mm]
НАВЧАЛЬНО-НАУКОВИЙ ІНСТИТУТ ТЕЛЕКОМУНІКАЦІЙ ТА ІНФОРМАТИЗАЦІЇ\\[2mm]
Кафедра комп'ютерної інженерії}

\vfill

\textbf{ЗАВДАННЯ \\[3mm]
на курсовий проект студенту\\[6mm]
Максімов Євгеній Сергійович\\[6mm]
}
\end{center}
\begin{enumerate}
	\item Тема проекту:
	\begin{enumerate}[label={1.\arabic*}]
		\item Мінімізація та побудова логічної функції.
		\item Побудова багатовходового дешифратора.
	\end{enumerate}
	\item Вхідні дані проекту:
	\begin{enumerate}[label={2.\arabic*}]
		\item Таблиця істинності 2-вох логічних функцій згідно варіанту.
		\item Таблиця обмежень побудови неповного дешифратора відповідно до варіанту для двох адресних просторів А1 і А2.
	\end{enumerate}
	\item Зміст розрахунково-пояснювальної записки (перелік питань, які потрібно
розробити):
	\begin{enumerate}[label={3.\arabic*}]
		\item  Перша логічна функція.
		\begin{enumerate}[label={3.1.\arabic*}]
			\item Мінімізація та побудова схеми функції.
			\item Переведення в базис І-НЕ(NAND) та побудова схеми функції.
			\item Переведення в базис АБО-НЕ(NOR) та побудова схеми функції.
		\end{enumerate}
		\item  Друга логічна функція.
		\begin{enumerate}[label={3.2.\arabic*}]
			\item Мінімізація та побудова схеми функцій.
			\item Переведення в базис І-НЕ(NAND) та побудова схеми функції.
			\item Переведення в базис АБО-НЕ(NOR) та побудова схеми функції.
		\end{enumerate}
		\item Побудова багатовходового дешифратора.
		\begin{enumerate}[label={3.3.\arabic*}]
			\item Побудувати схему неповного дешифратора відповідно до варіанту для двох адресних просторів А1 і А2.
			\item Привести  таблицю, в якій для кожної мікросхеми останнього ступеню вказати її адресний простір.
			\item Оцінити апаратні витрати на побудову дешифратора.
		\end{enumerate}
	\end{enumerate}
\end{enumerate}
\newpage
\begin{center}
\textbf{КАЛЕНДАРНИЙ ПЛАН\\[2mm]}
\begin{tabu} { | X[-2.5,с] | X[3,c] | X[0,c] | X[0,c] | }
 \hline
 № \newline з/п & Назва етапів виконання курсового проекту & Строк виконання етапів роботи & Примітка\\
 \hline
 1 & Підбір науково-технічної літератури &  &  \\
\hline
 2 & Проведення необхідних обчислень &  &  \\
\hline
 3 & Розробка  креслень&  &  \\
\hline
 4 & Розробка розрахунково-пояснювальної записки та реферату&  &  \\
\hline
\end{tabu}
\end{center}
\vspace{6mm}\\
\begin{flushright}
Студент
\begin{minipage}[t][2\mytextsize][t]{2in} % размер minipage равен удвоенному размеру основного шрифта
\underline{\hspace{2in}}\\ % линия подчёркивания на два дюйма
  \centering
  \small(підпис)
  \vspace{\mytextsize} % отступ minipage для выравнивания линии подчёркивания с базовой линией остального текста
\end{minipage} 
\begin{minipage}[t][2\mytextsize][t]{2in} % размер minipage равен удвоенному размеру основного шрифта
\underline{\hspace{2in}}\\ % линия подчёркивания на два дюйма
  \centering
  \small(прізвище та ініціали)
  \vspace{\mytextsize} % отступ minipage для выравнивания линии подчёркивания с базовой линией остального текста
\end{minipage}
\vspace{10mm}\\
Керівник роботи
\begin{minipage}[t][2\mytextsize][t]{2in} % размер minipage равен удвоенному размеру основного шрифта
\underline{\hspace{2in}}\\ % линия подчёркивания на два дюйма
  \centering
  \small(підпис)
  \vspace{\mytextsize} % отступ minipage для выравнивания линии подчёркивания с базовой линией остального текста
\end{minipage} 
\begin{minipage}[t][2\mytextsize][t]{2in} % размер minipage равен удвоенному размеру основного шрифта
\underline{\hspace{2in}}\\ % линия подчёркивания на два дюйма
  \centering
  \small(прізвище та ініціали)
  \vspace{\mytextsize} % отступ minipage для выравнивания линии подчёркивания с базовой линией остального текста
\end{minipage}
\end{flushright}
\newpage
\begin{center}
\textbf{РЕФЕРАТ}
\end{center}
\emph{Текстова частина курсового проекту:  26 с., 8 рис., 6 таб., 3 крес., 7 джерел.}

\textbf{Мета роботи} – Оптимізація логічних функцій та преведення їх до КМОН-логіки;Розробка багатоступеневого цифрового пристрою на основі існуючих мікросхем виконуючи задані обмеження.

\textbf{Методи дослідження} – статистичний метод, математичний метод.

Робота складається з 3-рьох основних частин:

\begin{itemize}
\item Теоретичні відомості використанні при воконанні роботи.

Наведені основні терміни та поняття для розуміння завдання, а також наведені необхідні математичні формули для проведення статистичних обчислень.
\item Виконане перше завдання курсового проекту з результатами логічних операцій та необхідними кресленнями.
\item Виконане друге завдання курсового проекту з результатами математичних, статистичних обчислень та необхідними кресленнями.
\end{itemize}
\newpage
\tableofcontents
\newpage
\chapter{Вступ}
\section{Теоретичні відомості до першого завдання}

\section{Теоретичні відомості до другого завдання}

\chapter{Перше завдання}
\section{Перша функція}

\subsection{Мінімізація та побудова схеми функції}

\subsection{Переведення в базис І-НЕ(NAND) та побудова схеми функції}
 
\subsection{Переведення в базис АБО-НЕ(NOR) та побудова схеми функції}
 
\section{Друга функція}

\subsection{Мінімізація та побудова схеми функції}

\subsection{Переведення в базис І-НЕ(NAND) та побудова схеми функції}
 
\subsection{Переведення в базис АБО-НЕ(NOR) та побудова схеми функції}

\chapter{Друге завдання}
\section{Характеристики дешифратора}

\subsection{Таблиця адресних просторів та схема неповного дешифратора}

\section{Апаратні витрати на побудову дешифратора}

\chapter{Висновок}

\newpage
\begin{thebibliography}{99}
\bibitem{DSTU_2394} \textit{ДСТУ 2394-94} Інформація та документація. Комплектування фонду,бібліографічний опис, аналіз документів. Терміни та визначення. — Чинний від
01.01.1995. — Київ: Держстандарт України, 1994. — 88 с.
\bibitem{DSTU_5034:2008} \textit{ДСТУ 5034:2008 } Інформація і документація. Науково-інформаційна діяльність.
Терміни та визначення понять. — Київ: Держспоживстандарт України, 2009. — 38 с.
\bibitem{DSTU_3008-95} \textit{ДСТУ 3008-95} Документація.
Звіти у сфері науки і техніки.Структура і правила оформлення.— Київ: Держспоживстандарт України,1995. - 39 c.
\bibitem{Frike_DE} \textit{Фрике К.} Вводный курс цифровой электроники — Москва:
Техносфера,2003. - 432 с. ISBN 5-94836-015-6
\bibitem{Tanenbaum_CA} \textit{Таненбаум Э.} Архитектура компьютера. 5-е изд. (+CD). — СПб.: Питер, 2007. — 844 с: ил. ISBN 5-469-01274-3 
\bibitem{metodichka}\textit{Торошанко Я.} Методичні рекомендації до курсового проекту. -
Київ : ДУТ, 2018. - 4 с.
\bibitem{Vernadskii}\textit{Національна бібліотека України імені В. І. Вернадського} [Електронний
ресурс]: [Веб-сайт]. – Електронні дані. – Київ : НБУВ, 2013-2017. – Режим доступу:
www.nbuv.gov.ua дата звернення 30.03.2017) – Назва з екрана.
\end{thebibliography}


\end{document}
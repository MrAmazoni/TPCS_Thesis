\begin{center}
\textbf{РЕФЕРАТ}
\end{center}
\emph{Текстова частина курсового проекту:  26 с., 8 рис., 6 таб., 3 крес., 7 джерел.}

\textbf{Мета роботи} – Оптимізація логічних функцій та преведення їх до КМОН-логіки;Розробка багатоступеневого цифрового пристрою на основі існуючих мікросхем виконуючи задані обмеження.

\textbf{Методи дослідження} – статистичний метод, математичний метод.

Робота складається з 3-рьох основних частин:

\begin{itemize}
\item Теоретичні відомості використанні при воконанні роботи.

Наведені основні терміни та поняття для розуміння завдання, а також наведені необхідні математичні формули для проведення статистичних обчислень.
\item Виконане перше завдання курсового проекту з результатами логічних операцій та необхідними кресленнями.
\item Виконане друге завдання курсового проекту з результатами математичних, статистичних обчислень та необхідними кресленнями.
\end{itemize}